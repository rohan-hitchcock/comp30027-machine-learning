\documentclass[11pt]{article}
\usepackage{colacl}
\usepackage{graphicx}
\usepackage[utf8]{inputenc}
\usepackage[dvipsnames]{xcolor}
\usepackage[inline]{enumitem}
\sloppy

%used for identifying drafting or notes
\newcommand{\drafting}[1]{\textcolor{OliveGreen}{#1}}

\title{COMP30027 Assignment 2 Report}
\author
{Anonymous}

\renewcommand\labelenumi{(\theenumi)}

\begin{document}
\maketitle

\section{Introduction}
The goal of this project was to predict the star ratings of a review of a restaurant based on the review text. The data available for learning and evaluation consisted of approximately 250000 Yelp reviews, and is made available by \cite{medhat_sentiment_2014} and \cite{rayana_collective_2015}. Each review consists of the review text, a rating (1, 3 or 5 stars) and review metadata (which was not used). We sought to investigate to what extent it was possible to predict the rating of a review from the review text alone. 

\section{Background}
This task is a type of sentiment analysis: the task of predicting sentiment from text. \drafting{discussion of background literature}

\section{Method} \label{sec:method}
We investigated models based on logistic regression (detailed in Section \ref{subsec:method-lr}) and support vector machine (detailed in Section \ref{subsec:method-svm}) techniques, in each case using the implementations made available through the Scikit Learn Python library \cite{sklearn_pedregosa_scikit-learn_2011}. These models used paragraph vector encodings of the review text, which was produced using the implementation made available through the Gensim Python library \cite{gensim_rehurek_software_2010}. 

\drafting{Move to results?}
In addition to model-specific hyperparameters (which will be discussed in the respective sections) the dimension of the parameter vector space embedding is a hyperparameter for all of our models. This can be interpreted as a measure of complexity of the model. As the dimension of the paragraph vector encoding a review increases, the degree to which the paragraph vector can capture the information in the review will increase too. Not all of the information in a review is likely to be useful for this classification task however\footnote{Review text also may contain information such as the cuisine of the restaurant, the time-of-day of the visit and the gender of the wait-staff, much of which is likely irrelevant to the task at hand.}, so there is a point at which increasing the dimension of the feature space will lead to over-fitting and degrade the performance of the classifier.
\subsection{Feature Selection} \label{subsec:method-features}

We investigated two text features, a Bag-of-Words model and paragraph vectors. We used Principle Component Analysis and Singular Value Decomposition to reduce the dimensionality of each of the pre-computed text features to 2 dimensions to help visualise any clusters or separation of classes. It was found that the high dimensional paragraph vectors did a much better job of separating the instances of each class than Bag-of-words models, as unique clusters were more visible and separated.

Further testing was done to see if a combination of attributes from these features (ie, words from the Bag-of-Words model added to paragraph vectors of increasing dimensions) could lead to an improvement in accuracy than either one of them alone when used to train a basic Logistic Regression model. We found that Paragraph vectors performed far better, and the addition of the K-best words from the Bag-of-Words model (Selected using Chi2), \drafting{(I have actually got an explanation for why, should I put it in?)} to a paragraph vector feature set had no significant impact on performance.

We decided to use Paragraph Vectors as our primary text feature for further analysis.
\subsection{Logistic regression} \label{subsec:method-lr}
\drafting{
We investigated a single Logistic Regression model for multi-class classification; Multinomial Logistic Regression. Multinomial Logistic Regression generalises Logistic Regression to multi class problems. It is probabilistic, constructing a linear predictor function that assigns a score to each instance by computing the dot-product of it's features with their corresponding weights. We choose the weights to have the maximum likelihood to observing all training instances with their particular classes. New instances are classified by determining which class is most likely given its features.
Multinomial Logistic Regression has a single hyperparameter C, which is the inverse of the L2-Regularisation strength, a measure of how strict the model is fit to the training data and avoids mis-classifying training instances.
We investigated several ensembles of Logistic Regression to determine whether the performance could be improved, using Boosting and Bagging techniques.
Paragraph Vectors were used over a Bag-of-words as explained in section ____}
\subsubsection*{Justifying Logistic Regression model selection}
\drafting{
Other options exist for multi-class classification using Logistic Regression, such as training multiple binary Logistic Regressors (one for each class label) using a One-versus-rest scheme and predicting based on the model with the highest score, or a popular alternative Ordinal Logistic Regression which is particularly suited for ordinal class labels.
One-versus-rest models are less efficient, tending to have larger standard errors (Agresti, 2002).
Ordinal Logistic Regression is not implemented as part of the standard Scikit Learn Python library and thus was not investigated due to time constraints.
}

\drafting{
Another important note is that Multinomial Logistic Regression assumes the Independence of Irrelevant Alternatives axiom. It is an axiom of Decision Theory and other social sciences, which varies in definition depending on context. In social sciences, it can be thought of as meaning:
`The social preferences between alternatives x and y depend only on the individual preferences between x and y' (Arrow, 1963).
We can relate this to the context of reviewing restaurants by saying:
Given a reviewer gives a rating of 1 to a particular restaurant, with options only 1 or 5, the inclusion of a third option 3, should not make the alternative original option 5 more preferable (It may however make the new option, 3, more preferable).
We think this assumption is acceptable given the motivation behind choosing a particular rating for a restaurant is presumed to be based on an experience which is unchanged after the fact.
}

\drafting{
Logistic Regression was chosen as a probabilistic model as a point of comparison to a linear model such as SVM providing a variety of model types for classification.
}

\drafting{
Logistic Regression may perform better as a probabilistic supervised Machine learning classifier than the more commonly used Naive Bayes, as it does not make the assumption that features are conditionally independent. Naive Bayes has higher bias but lower variance than Logistic Regression, and in general, as the training size tends to infinity, Logistic Regression performs better as a classifier (Ng and Jordan, 2002)
}}

\subsection{Support vector machine} \label{subsec:method-svm}
We investigated three variants of support vector machine (SVM) classifiers: a SVM with a linear kernel and one-verses-rest multi-class classification (Linear-SVM), a SVM model with a radial basis function (RBF) kernel and one-verses-rest multi-class classification (RBF-SVM), and a SVM with a linear kernel separating the positive and negative sentiment classes, in which marginal instances are classified as having neutral sentiment (Binary-SVM). 

All three SVM-based model have a regularisation coefficient $C$ as a hyperparameter, which determines the trade-off between margin maximisation and training error minimisation. As $C$ increases the margin is increased, however more training instances are allowed to be misclassified. The RBF-SVM also has the kernel hyperparameter $\gamma$, which in this case was used to scale $\frac{1}{\textrm{Var}(X) \cdot numfeatures}$ (\drafting{explain this better}). The Binary-SVM has the probability threshold hyperparameter, which is the probability a negative of positive sentiment prediction must exceed to be classified as that class \drafting{awkward wording here}.

\subsubsection*{Justifying SVM model selection}
Paragraph vector encoding is well suited to classification by an SVM model since paragraph vectors have meaningful geometric relationships to one another \cite{le_distributed_2014}, and SVM models attempt to exploit the geometry of the feature space by fitting a separating hyperplane. For example, the relative distance between points in the paragraph vector space is relevant to their relative meaning. This is in contrast Bag-of-Words vectors, in which the vector encodings of the (one word) texts ``horrible'', ``terrible'' and ``great'' are equidistant in the Bag-of-Words feature space.

The paragraph vector encoding of text is also designed to transfer relationships in the meaning of text to linear relationships between paragraph vectors: ``king'' - ``man'' + ``woman'' = ``queen'' \cite{le_distributed_2014}. If we suppose that negative and positive sentiment reviews have exactly opposite meanings (with neutral sentiment reviews somewhere in between) then we would expect the review text of these classes to be linearly separable with the paragraph vector encoding. Preliminary testing of different kernel functions (without tuning any hyperparameters) indeed showed that an SVM classifier with a linear kernel performed the best in every metric, however an SVM classifier came quite close in performance. For this reason we chose to continue with extended analysis of both.

Under the assumption that positive and negative sentiment reviews are exactly opposite in meaning and that neutral sentiment reviews are some mixture (i.e. a linear combination in paragraph vector space) of positive and negative meanings, it seems plausible that the neutral sentiment reviews will lie close to the hyperplane which separates the positive and negative sentiment reviews. This encourages the consideration of a binary SVM classifier which distinguishes positive and negative reviews, where instances which fall in the margin are classified as neutral. Rather than using the raw distance to the hyperplane, opted to use a probabilistic variant of SVM which estimates the probability of an instance belonging to each class based on its distance to the hyperplane \cite{platt_probabilistic_1999}. This allowed us to use a probability threshold for neutral classification as a hyperparameter, rather than a distance threshold, which means the value of this threshold is independent of the specific paragraph vector embedding\footnote{The specific embedding of a particular text is only relevant in the context of the training corpus used to compute the embedding function \cite{le_distributed_2014}.}. A probability threshold also has the benefit of making the model more interpretable. Despite its apparent theoretical foundations, results in the literature suggest that such a classifier will always result in worse performance as compared to using a standard multi-class approach \cite{koppel_importance_2006}. Nevertheless we chose to pursue this approach as well.


\subsection{Stacking model}
\drafting{
Stacking aims to increase predictive performance by training base classifiers on the training data, then training a meta classifier on the outputs of the base classifiers as features. We investigated stacking all pairs from our best classifiers, Logistic Regression, RBF SVM and Linear SVM, as well as all three. For every combination, we used the parameters found to produce the best performance for each classifier alone. Logistic Regression was used as the meta-classifier in each case (with no parameter tuning).
We further analysed the performance of the best stacked classifier in detail.
}

\section{Results}
\drafting{discussion of why we used stratified splits?}
In order to identify the optimal feature space dimension for each model, we first evaluated our model \drafting{(using what metric?)} (with hyperparameters untuned) on paragraph vector of dimensions 25 to 300 in steps of 25 encodings using an 80:20 stratified random holdout, only using the training set in each case to compute the paragraph vector embedding function. A random holdout method, rather than cross-validation, was used due to the computation expense of computing paragraph vector encodings: producing cross-validation splits for many dimensions was out of the question. For the \drafting{logistic regression}, RBF-SVM and Binary-SVM models this dimension was 125, and for the Linear-SVM model this dimension was 150. 

We then computed the paragraph vector encodings for a 5-fold stratified cross validation split at that dimension, again only using the training set of each split to produce the encoding. This pre-computed split was then used to tune the remaining hyperparameters. Since each cross validation run \drafting{took a long time} parameter values were initially adjusted in large increments, and then a finer full grid search was done on regions of interest. The results of this can be seen in Figure \drafting{insert grid search figure}.  \drafting{What are the best hyperparameters? with reference to the figures} \drafting{Why cross validaton? (to prevent overfitting the hyperparameters)} 

After tuning the hyperparameters, we evaluated the final model trained on paragraph vectors of dimensions from 25 to 300 in steps of 25 to produce the learning curve in Figure \drafting{insert figure of learning curves}. As discussed in Section \ref{sec:method} the dimension of the paragraph vector encoding is a measure of the complexity of our model. We also evaluated our model on the 5-fold cross validation split at the optimal dimension, the results of which can be seen in Figure \drafting{insert figure of confusion matrix, and any other stuff we want to talk about in analysis}.

\drafting{I think the stacking model would have been treated differently, need to talk about this here.}

\drafting{
Describe and compare model performance. (i.e. just say what the figures show)
}



\section{Analysis}
\drafting{
Explain and interpret results. Identify weeknesses in methodology (more cross validation runs, more kernels for SVM).
include comparison to baseline classifier (Zero-R, we don't acually need to compute this can just look at class proportions), error analysis via confusion matrix (put the figure in the results section though)
}

\drafting{
    \begin{itemize}
        \item All models have best performance on feature vectors of a similar dimension. Suggest this is because this is the best dimension for this text corpus for representing the information relevent to the rating. It is a good thing this is independent of the model since\dots
        \item As expected the Binary-SVM performed worse than the SVM classifiers using a ovr approach to multi-class classification. This is consistent with literature. Note that is is a lot faster though.
        \item All classifiers cap at around the same performance accross all metrics, including the stacking classifier. It is possible that this is just the best that can be done with this feature representation of the review text. Also can discuss how it is not reasonable to expect perfect prediction even given a perfect representation of text features: the review score gives a (human) reader context for the sentiment of the review text. eg ``chips were good'' with a 1 and 5 star review gives very different impressions (maybe think of a better example).
    \end{itemize}
}

\section{Conclusion}
\drafting{Summarise main points and (possibly) future work.}

%put any citations here which must be included in the bibliography but won't necessarily be referenced

%citations for the dataset
\nocite{mukherjee_what_2013}
\nocite{rayana_collective_2015}

%citations for software libraries
\nocite{sklearn_pedregosa_scikit-learn_2011}
\nocite{gensim_rehurek_software_2010}
\bibliographystyle{acl}
\bibliography{report}

\end{document}
